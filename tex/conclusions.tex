\vspace{-0.2cm}
\section{Conclusions}\label{sec:conclusions}

The main contributions of this research is to evaluate the lattice-based KEM and potential NIST post-quantum standard, FrodoKEM \cite{frodokem}, in hardware. We develop designs which can reach up to 825 operations per second, where most of the designs fit in under 1500 slices. Area consumption results are less than the previous state-of-the-art, and are much lower than many of the other post-quantum hardware designs shown in Table \ref{tab:pqc}. We significantly improve the throughput performance compared to the state-of-the-art, by increasing the number of parallel multipliers we use during matrix multiplication. In order to do this efficiently, we replace an inefficient PRNG previously used, cSHAKE, with a much faster and smaller PRNG, Trivium. As a result, we are able to obtain either a much lower FPGA footprint (up to 5x smaller) or a much higher throughput (up to 16x faster) compared to previous research. Our implementations run in constant computational time and the designs comply with the Round 2 version of FrodoKEM in all aspects except for this PRNG choice. To further evaluate the performance of FrodoKEM, we implemented first-order masking for decapsulation, and we showed that it can be achieved with almost no effect on performance.

The results show that FrodoKEM is an ideal candidate for hardware designs, showing potential for high-throughput performances whilst still maintaining relatively small FPGA area consumption. Moreover, compared to other NIST lattice-based candidates, it has a lot more flexibility, such as increasing throughput without completely re-designing the multiplication component, compared to, for example, a NTT multiplier.
