\section{Introduction} \label{sec:Introdution}

The future development of a scalable quantum computer will allow us to solve, in polynomial time, several problems which are considered intractable for classical computers. Certain fields, such as biology and physics, would certainly benefit from this ``quantum speed up'', however this could be disastrous for security. The security of our current public-key infrastructure is based on the computational hardness of the integer factorization problem (RSA) and the discrete logarithm problem (ECC). These problems, however, will be solved in polynomial time by a machine capable of executing Shor's algorithm~\cite{Shor}.

To promptly react to the threat, the scientific community started to study, propose, and implement public-key algorithms, to be deployed on classical computers, but based on problems computationally difficult to solve also using a quantum or classical computer. This effort is supported by governmental and standardization agencies, which are pushing for new and quantum resistant algorithms. The most notable example of these activities is the open contest that NIST \cite{nistpq} is running for the selection of the next public-key standardized algorithms. The contest started at the end of 2017 and is expected to run for 5 to 7 years.

Approximately seventy algorithms were submitted to the standardization process, with the large majority of them being based on the hardness of lattice problems. Lattice-based cryptographic algorithms are a class of algorithms which base their security on the hardness of problems such as finding the shortest non-zero vector in a lattice. The reason for such a large number of candidates is because lattice-based algorithms are extremely promising: they can be implemented efficiently and they are extremely versatile, allowing to efficiently implement cryptographic primitives such as digital signatures, key encapsulation, and identity-based encryption. 

As in the past case for standardizing AES and SHA-3, the parameters which will be used for selection include the security of the algorithm and its efficiency when implemented in hardware and software. NIST have also stated that algorithms which can be made robust against physical attacks in an effective and efficient way will be preferred \cite{nistsca}. Thus, it is important, during the scrutiny of the candidates, to explore the potential of implementing these algorithms on a variety of platforms, and to assess the overhead of adding countermeasures against physical attacks.

To this end, this paper concentrates on FrodoKEM, a key encapsulation algorithm submitted to NIST as a potential post-quantum standard. FrodoKEM is a conservative candidate due to its hardness being based on standard lattices, as opposed to Ring-LWE or Module-LWE, as such it has had limited practical evaluations. Thus, we explore the possibility to efficiently implementing it in hardware and we estimate the overhead of protection against power analysis attacks using first-order masking. To maximize the throughput, we rely on a parallelised implementation of the matrix multiplication. We also aim to have a relatively low area consumption on the FPGA. To be parallelised, however, the matrix multiplication requires the use of a smaller and more performant pseudo-random number generator. We propose to achieve the performance required for the PRNG by using Trivium, an international standard under ISO/IEC 29192-3 \cite{ISO} and selected as part of the eSTREAM project. We utilize this instead of AES or SHAKE, as per the FrodoKEM specifications. We do this as a design exploration study, but, all the security arguments from the specifications still apply.

The rest of the paper is organized as follows. Section~\ref{sec:related} discusses the background and the related works. Section~\ref{sec:design} introduces the proposed hardware architectures and the main design decisions. Section~\ref{sec:results} reports the results obtained while synthesizing our design on re-configurable hardware and compares our performance against the state-of-the-art. We conclude the paper in Section \ref{sec:conclusions}.